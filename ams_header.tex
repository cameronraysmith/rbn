%!TEX root = paper.tex
%------------------------------------------------------------------------------
% Beginning of paper.tex
%------------------------------------------------------------------------------
%
% AMS-LaTeX version 2 sample file for journals, based on amsart.cls.
%
%        ***     DO NOT USE THIS FILE AS A STARTER.      ***
%        ***  USE THE JOURNAL-SPECIFIC *.TEMPLATE FILE.  ***
%
% Replace amsart by the documentclass for the target journal, e.g., tran-l.
%
\documentclass{amsart}

%     If your article includes graphics, uncomment this command.
\usepackage{graphicx}

\newtheorem{theorem}{Theorem}[section]
\newtheorem{lemma}[theorem]{Lemma}

\theoremstyle{definition}
\newtheorem{definition}[theorem]{Definition}
\newtheorem{example}[theorem]{Example}
\newtheorem{xca}[theorem]{Exercise}

\theoremstyle{remark}
\newtheorem{remark}[theorem]{Remark}

\numberwithin{equation}{section}

%    Absolute value notation
\newcommand{\abs}[1]{\lvert#1\rvert}

%    Blank box placeholder for figures (to avoid requiring any
%    particular graphics capabilities for printing this document).
\newcommand{\blankbox}[2]{%
  \parbox{\columnwidth}{\centering
%    Set fboxsep to 0 so that the actual size of the box will match the
%    given measurements more closely.
    \setlength{\fboxsep}{0pt}%
    \fbox{\raisebox{0pt}[#2]{\hspace{#1}}}%
  }%
}

% user added
\usepackage{longtable}
\usepackage{dot2texi}
\usepackage{tikz}
\usetikzlibrary{automata,shapes,arrows}
% todo notes see http://www.texample.net/tikz/examples/todo-notes/
\usepackage[colorinlistoftodos]{todonotes}
\usepackage{hyperref}
\hypersetup{colorlinks=true,
linkcolor=[rgb]{.67 .27 .27},
citecolor=[rgb]{.09 .29 .54},
urlcolor=[rgb]{.09 .29 .54}}
\usepackage{placeins}

% Set autoref text
% http://tex.stackexchange.com/a/36576/6784
\renewcommand*{\figureautorefname}{Fig.}
\renewcommand*{\equationautorefname}{Eq.}
\renewcommand*{\tableautorefname}{Table}
\renewcommand*{\sectionautorefname}{Section}


\def\tr{\mathrm{tr}}
\def\Path{\mathrm{Path}}
\def\hier{\mathrm{Hier}}
\def\Vertex{\mathrm{Vertex}}
\def\adj{\mathrm{adj}}

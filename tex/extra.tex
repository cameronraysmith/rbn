\section{Stability criteria}
% \todo{Since there is no point in publishing proofs of results contemporary with Darwin's Origin of Species, this section will be replaced with a reference to Gantmacher.}
Suppose that $M$ is an $n \times n$ matrix with real coefficients.
As previously described, we call
the matrix $M$ stable if all its eigenvalues have negative real part.
Define the characteristic polynomial as $\chi(M)(z) = \det(zI - M)$.
Denote the coefficients of $\chi(M)$ as $s_j$ and its roots as $r_j$,
writing $p(x) = x^n + \sum_{k=1}^n s_k x^{n-k} = \prod_{k=1}^n (x-r_k)$.
Then, since the roots of $\chi$ are the eigenvalues of $M$, asking that
$M$ be stable is equivalent to asking that the roots of $\chi$ all have
negative real part.  Hence, let us call a polynomial stable if all its
roots have negative real part.  We will formulate conditions for this to
happen in terms of the coefficients of $\chi$.

\begin{lemma}
If $M$ is stable, then $s_i > 0$ for all $i$ between $1$ and $n$.
\end{lemma}

\begin{proof}
Assume that $M$ is stable and that it has $2m$ complex roots which come in
complex conjugate pairs and $n-m$ real roots.  Let $r_1 \ldots, r_{2m}$ be
the complex roots with $r_{2k+1}$ being the complex conjugate of $r_{2k}$
and let $r_{2m+1}, \ldots,  r_{n}$ be the real roots. Then we can write the
factorization as follows:
\[
\chi(M)(z) = \prod_{k=1}^{m} (z^2 - (r_{2k} + r_{2k+1}) z + r_{2k} r_{2k+1})
             \prod_{k=2m+1}^{n} (z - r_k)
\]
Since $r_{2k}$ and $r_{2k+1}$ are complex conjugates, $r_{2k} + r_{2k+1} =
\Re r_{2k} = \Re r_{2k+1}$ and $r_{k} r_{2k+1} > 0$.   Since all the
roots have negative real part, all the coefficients of the terms in
the products above are positive so, when we multiply out the product,
we will have all positive terms.  Hence, we conclude that a necessary
condition for $M$ to be stable is that $s_k > 0$ for all $k$.
\end{proof}

To state the next criterion, we will require the quantity $\Delta$, which
is defined as $\prod_{j=1}^{n-1} \prod_{k=j+1}^n (r_1 + r_j)$.  Since
this expression is invariant under permuting the roots, it may be expressed
as a function of the coefficients $s_i$; for small values of $n$, the
resulting expression looks as follows:
\begin{align*}
n=2 \qquad &\Delta = s_1 \\
n=3 \qquad &\Delta = s_3 - s_1 s_2 \\
n=4 \qquad &\Delta = s_1 s_2 s_3 - s_3^2 - s_1^2 s_4 \\
n=5 \qquad &\Delta = s_1 s_2 s_3 s_4 - s_3^2 s_4 - s_1^2 s_4^2 -
                     s_1 s_2^2 s_5 + s_2 s_3 s_5 + 2 s_1 s_4 s_5 - s_5^2
\end{align*}

\begin{lemma}
If a point lies on the boundary of the set of stable polynomials, then
either $s_n = 0$ and $\Delta = 0$.
\end{lemma}

\begin{proof}
Let $S_n$ denote the symmetric group on $n$ objects and let $f \colon
\mathbb{C}^n/S_n \to \mathbb{C}^n$ be the map that sends a set of roots
to the n-tuplet of coefficients of a polynomial.  This map is explicitly
given by the elemetary symmetric polynomials.  By the fundamental theorem
of algebra, it is invertible and, by ???, its inverse is continuous.

Suppose that $p$ is a stable polynomial.  By definition, this means that
$\Re r > 0$ for every root $r$ of $p$.  Since the set $\{r \mid \Re r > 0\}$
is an open subset of $\mathbb{C}^n/S_n$, the continuity of $f^{-1}$ implies
that the set of stable polynomials is open.  Likewise, the set of unstable
polynomials none of whose roots have real part equal to zero is open.

Hence, if a point lies on the boundary of the stable region, it must
correspond to a polynomial with at least one root whose real part is zero.
There are two ways this could happen.  If the root is real, then having
zero real part is the same as the root being zero, hence $s_n = 0$.  If
the root is complex, then having zero real part means that the root and
its complex conjugate are purely imaginary and sum to zero, hence $\Delta = 0$.
\end{proof}

\begin{lemma}
If $n < 6$, then a polynomial $p$ is stable if and only
if $\Delta > 0$ and $s_1 > 0$ for all $i$.
\end{lemma}

\begin{proof}
Suppose that, to the contrary, there exists a polynomial which satisfies
the inequalities $\Delta > 0$ and $s_1 > 0$ but which is not stable.
Consider a straight line connecting this polynomial to a stable polynomial,
say $(x - 1)^n$.

Suppose that $p$ is a polynomial for which $\Delta = 0$ and $s_i > 0$ but
which is not on the boundary of the stable region.  In order to have
$\Delta = 0$, we must either have a pair of real roots which sum to zero
or a pair of complex conjugate roots with zero real part.  In the former
case, we would have $x^2 - r^2$ as a factor of our polynomial and in the
latter, we would have $x^2 + r^2$ as a factor.

Suppose that $x^2 - r^2$ is a factor.  Then we have
\[
p = (x^2 - r^2) \left(x^{n-2} + \sum_{k=1}^{n-3} c_k x^{n-k-2} \right) =
x^n + \sum_{k=3}^{n-3} (c_k - r^2 c_{k-2}) x^{n-k} +
\]

Suppose that $x^2 + r^2$ is a factor. Then we have one of the following
when $2 < n < 6$
\begin{align*}
(x^2 + r^2)(x + a_1) &=
x^3 + a_1 x^3 + r^2 x^2 + a_1 r^2 \\
(x^2 + r^2)(x^2 + a_1 x + a_2) &=
x^4 + a_1 x^3 + (a_2 + r^2) x^2 + a_1 r^2 x + a_2 r^2 \\
(x^2 + r^2)(x^3 + a_1 x^2 + a_2 x + a_3) &=
x^5 + a_1 x^4 + (a_2 + r^2) x^3 + (a_3 + a_1 r^2) x^2 + a_2 r^2 x + a_3 r^2
\end{align*}
Inspecting these, we see that, if the coefficients of $p$ are positive,
then $a_i > 0$ for all $i$.

$\ldots$
\end{proof}

For $n = 6$, this argument breaks down because, for instance, we have
the polynomial $x^6 + x^5 +\frac{1}{2} x^4 + 2x^3 + \frac{1}{2} x^2 +
x + 1$  which factors as  $(x^2 + 1)(x^4 + x^3 - \frac{1}{2} x^2 + x + 1)$.
This polynomial has positive cofficients and $\Delta = 0$ but does not
lie on the boundary of the stable region because the second factor has
a negative coefficient, hence has a root with positive real part.  Thus,
for $n = 6$ and higher, we need to supplement our conditions with
additional inequalities (which may be taken to be linear) to exclude
such possibilitiies.  However, we will not pursue this topic further
in this work because we will only be making computations with matrices
of size less than $6$.

\section{Paths and eigenvalues}
% \todo{I am thinking that we might want to jettison this section from
% the published version and leave it as private notes for future work.
% As a substitute for the published version, I made a section in front
% of it which takes a more direct route to proving what is needed for
% the purposes of showing that the characteristic polynomial factors and
% making our arguments about total orders being the most stable.  We
% should discuss this point tomorrow.}

Next, we will want to relate prpoperties of our matrix and its eigenvalues
to  properties of the associated graph $G_A$.  To do this, we will first
examine the category of paths on $G_A$, and derive a formula expressing
the chararacteristic polynomial as a sum over cycles.

Given a digraph $G$, let $\Path (a,b)$ denote the set of paths from the
vertex $a$ to the vertex $b$ and let $\Path(G) = \bigcup_{a,b \in G} \Path (a,b)$.
Given a path $p_1$ from $a$ to $b$ and a path $p_2$ from $b$ to $c$, we may
concatenate them to obtain a path $p_1 \circ p_2$ from $a$ to $c$.  Since
this operation is associative and has identity elements,
$(\Vertex(G), \Path(G), \circ)$ forms a category.  Define
$\mathcal{A}(G)$ to be the set of all linear combinations of elements of
$\Path(G)$ with coefficients in $\mathbb{R}$.  Then $\mathcal{A}(G)$ forms
an associative algebra under the multiplication induced by $\circ$.

Next, we introduce two basic functors from $\Path (G)$.  One of these is
$\length \colon \Path(G) \to \mathbb{N}$ (regarding the natural numbers
as a category with a single object) which sends each path to its length.
The other functor is the weight product functor.  Given an assignement
$w \colon \Path(G) \to \mathbb(R)$ of weights to the edges of our digraph,
define $\wprod \colon \Path(G) \to \mathbb{R}$ to be the function which sends
each path to the product of the weights associated to the edges which
that path traverses.  Given the way that graphs are associated to matrices,
weights on edges correspond to matrix entries and, given a path
$v_1, v_2, v_3 \ldots$, we have $\wprod(v_1, v_2, v_3 \ldots) =
m_{v_1 v_2} m_{v_2 v_3} \cdots$.  Going in the opposite direction, we
note that any functor from $\Path(G)$ to $\mathbb{R}$ may be regarded as
a weighted product for a suitable choice of weights.  We extend $\wprod$ to
$\mathcal{A}(G)$ by linearity; the result is an algebra homomorphism.

% \todo{Start talking about Abelianizing the category and the algebra, say
% more after decomposing into components.}

% \todo{Characterize strongly connected components in terms of
% fibration over poset.}

The characteristic polynomial $\chi(A)$ associated to $A$ can be factored
according to the strongly connected components of the graph $G$.
\[
\frac{\chi(M)'(z)}{\chi(M)(z)} = \tr (zI - M)^{-1}
\]
where, $\chi(M)(z) = \det (zI - M)$ is the chararacteristic polynomial of the
matrix $M$ with $z$ a complex variable.  This formula is readily verified when
$M$ is diagonal and has the nice property of being additive over decompositions
of the matrix. Making the change of variables $x=\frac{1}{z}$, we have
\[
\tr (I - xM)^n =
\sum_{a \in \Vertex(G)} \sum_{p \in \Path_n (G)(a,a)} \wprod(p) x^{\length(p)}
\]
where $G$ is the graph of the matrix and $\Path_n (G;a,b)$ is the set of
paths of length $n$ from $a$ to $b$.  ; $F$ is the functor from this category
to the multiplicative semigroup of the real numbers which corresponds to the matrix $M$.

As an illustration of this formula, we will use it to compute the characteristic
polynomial of the matrix associated to the labelled graph in figure ??.  We begin
by noting that the closed paths of this graph can be described by the following
regular expressions:
\begin{align*}
\Path (1,1) &: (e + bd + bca)^* \\
\Path (2,2) &: (ae^*(be^*d)^*bc)^* \\
\Path (3,3) &: ((ca + d)e^*b)^* \\
\end{align*}
Since these regular expressions are non-redundant, we can obtain generating
functions for the paths by considering $a,b,c,d,e$ as non-commutative
variables and replacing the Kleene star $x^*$ by $(1 - x)^{-1}$:
\begin{align*}
\Path (1,1) &: (1 - e - bd - bca)^{-1} \\
\Path (2,2) &: (1 - a(1-e)^{-1}(1 - b(1-e)^{-1}d)^{-1}bc)^{-1} \\
\Path (3,3) &: (1 - (ca + d)(1 - e)^{-1}b)^{-1} \\
\end{align*}
To obtain the characteristic polynomial, we make the substitutions
$a \mapsto m_{12}/z, b \mapsto m_{31}/z, c \mapsto m_{23}/z, d \mapsto m_{13}/z,
e \mapsto m_{11}/z$ and simplify, using the fact that the matrix entries
commute:
\begin{align*}
\Path (1,1) &: \frac{z^3}
{z^3 - m_{11} z^2 - m_{13} m_{31} z - m_{12} m_{23} m_{31} } \\
\Path (2,2) &: \frac{z(z^2 - m_{11} z - m_{13} m_{31})}
{z^3 - m_{11} z^2 - m_{13} m_{31} z - m_{12} m_{23} m_{31} }  \\
\Path (3,3) &: \frac{z^2(z^2 - m_{11})}
{z^3 - m_{11} z^2 - m_{13} m_{31} z - m_{12} m_{23} m_{31} } \\
\end{align*}
Adding these three quantitites together, we obtain
\[
\frac{z(3 z^2 - 2 m_{11} z - m_{13} m_{31})}
{z^3 - m_{11} z^2 - m_{13} m_{31} z - m_{12} m_{23} m_{31} },
\]
which we recognize as $\chi'(z)/\chi(z)$, where
$\chi(z) = z^3−m_{11} z^2 - m_{13} m_{31} z - m_{12} m_{23} m_{31}$.
finally, we note that this polynomial is indeed the characteristic
polynomial of the matrix
\[
 \begin{pmatrix}
   m_{11} & m_{12} & m_{13} \\
   0      & 0      & m_{23} \\
   m_{31} & 0      & 0 \\
 \end{pmatrix}
\]
corresponding to our digraph.

We now make use of the additivity.  Let $\hier$ be the functor which
sends $G$ to a poset $P$.  Then every element $p$ of $\Path(a,a)$ lies
in $\hier^{-1} (\hier(a))$.  Hence, our sum decomposes into partial sums
for the fibers of the points in the hierarchy which corresponds to a
factorization of the characteristic polynomial.

\section{Bounds on robustness}

Suppose that $S$ is a region in $[0,1]^2$.  Then, the robustness of $S$ under resampling is given as
\begin{align*}
R &= \frac{\int_0^1 dx \int_0^1 dy \int_0^1 dx'\, \mathbf{1}_{S} (x,y) \mathbf{1}_{S} (x',y) +
           \int_0^1 dx \int_0^1 dy \int_0^1 dx\,dy\,dy'\, \mathbf{1}_{S} (x,y) \mathbf{1}_{S} (x,y')}
          {2 \int_0^1 dx \int_0^1 dx\,dy\, \mathbf{1}_{S} (x,y)} \\
  &= \frac{\int_0^1 dy\, (\int_0^1 dx\, \mathbf{1}_{S} (x,y))^2 +
           \int_0^1 dx\, (\int_0^1 dy\, \mathbf{1}_{S} (x,y))^2}
          {2 \int_0^1 dx\, \int_0^1 dx\,dy\, \mathbf{1}_{S} (x,y)}
\end{align*}
We will show that $R \le \int_{H^2} dx\,dy\, \mathbf{1}_{S} (x,y)$ with equality occurring when $S$ is rearrangement equivalent to either the rectangle of height 1 or a rectangle of width 1.

First, we start with the special case when $S$ is a rectangle with one vertex at the origin and sides parallel to the coordinate axes.  If $a$ is the height of our rectangle and $b$ is its width, we have

Next, consider the case where $R$ consists of a union of squares of a rectangular grid obtained by subdividing the $x$ and $y$ axes of our bounding square into $2N$ portions of equal length.  Given two integers $i$ and $j$

Suppose we restrict to rectangular regions in an $n$-dimensional hypercube and resample $m$ variables.  Then we have $R = s_m (a_1, \ldots a_n)$ where $s_m$ is the $m$-th elementary symmetric polynomial.  As before, we have the constraint that $\prod_i a_i = s_n (a_1, \ldots a_n)$ is constant.  Using Lagrange multipliers, we obtain the equation
\[
\frac{\partial s_m (a_1, \ldots a_n)}{a_k} = \lambda \frac{\partial s_n (a_1, \ldots a_n)}{a_k} .
\]
Since the elementary symmetic polynomial is linear in each of its variables, this reduces to
\[
s_{m-1} (a_1, \ldots {\hat a}_k \ldots a_n) = \lambda \partial s_{n-1} (a_1, \ldots {\hat a}_k \ldots a_n) .
\]
Dividing through, this becomes
\[
s_{m-1} \left(\frac{1}{a_1}, \ldots {\hat a}_k \ldots \frac{1}{a_n}\right) = \lambda
\]
or, eliminating $\lambda$, we have
\[
s_{m-1} \left(\frac{1}{a_1}, \ldots {\hat a}_j \ldots \frac{1}{a_n}\right) =
s_{m-1} \left(\frac{1}{a_1}, \ldots {\hat a}_k \ldots \frac{1}{a_n}\right)
\]
for all $j,k$.  Suppose that $j < k$.  Then, we may expaand both sides,
\begin{align*}
&\frac{1}{a_k} s_{m-2} \left(\frac{1}{a_1}, \ldots {\hat a}_j \ldots {\hat a}_k \ldots \frac{1}{a_n}\right) +
s_{m-1} \left(\frac{1}{a_1}, \ldots {\hat a}_j \ldots {\hat a}_k \ldots \frac{1}{a_n}\right) = \\ \qquad
&\frac{1}{a_j} s_{m-2} s_{m-1} \left(\frac{1}{a_1}, \ldots {\hat a}_j \ldots {\hat a}_k \ldots \frac{1}{a_n}\right) +
s_{m-1} \left(\frac{1}{a_1}, \ldots {\hat a}_j \ldots {\hat a}_k \ldots \frac{1}{a_n}\right)
\end{align*}
and conclude that $a_j = a_k$.  Hence, the only critical point lies at the point $a_1 = a_2 = \cdots = a_n$.  By the second derivative test, that point is a minimum, so the maximum must lie on the boundary.

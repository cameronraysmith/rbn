%!TEX root = ../paper.tex
Here we determine the relationship between network hierarchy, a topological property, and the probability of \emph{robustness} or \emph{structural stability}, a dynamical property \cite{Smale1967}. Robustness is of interest at all scales of the biological hierarchy, and has been previously studied in the context of gene-regulatory networks \cite{WADDINGTON1942a,VanNimwegen1999,Siegal2002,Ciliberti2007b,Ciliberti2007,Wagner2013}. Intuitively, robustness is the probability of stability to perturbation in the system structure or its parameters for systems which have already been determined to be stable in the sense of linear stability analysis \cite{Davis1962}.

The dynamical model given in terms of a system of differential equations for any network can be represented in terms of an interaction graph (\ref{fig:biomodelexamples} top row). These interaction graphs can be viewed as deriving from the combination of system modules that accept a given pattern of inputs and produce a given pattern of outputs (\reffigexamplesystemmodules). Symmetries are characterized by the ability to interchange these modules or their connectivity without changing some property of the system or its dynamics.

The network architecture can be represented in terms of an adjacency matrix and further abstracted by mapping the interaction graph to the network of strongly connected components (SCCs, see Supplementary Information) (\reffigscc). This map from the interaction graph of a biological network, referred to as $\hier$, has a collection of symmetries shown in \reffighiertransformations. These three symmetries represent transformations that can be performed on the interaction graph that do not change the network of SCCs to which it is associated \reffigscc. Two of these three are also symmetries with respect to dynamical robustness. \ref{fig:robustnesssymmetries} shows an example of these symmetries applied to a specific interaction graph.

We have derived an analytical expression for dynamical robustness, $R_{tot}$, of a biological network in terms of its interaction graph as a weighted average of the robustness of the SCCs, $R_i$, the corresponding number of links within each SCC, $d_i$, and the number of links between the SCCs, $l$. This expression is fully developed in Supplementary Information \ref{eq:sccrobustness}, but can be schematized as in \ref{eq:robschematic} (see \reffigscc$\,$ for examples demonstrating this expression)
\begin{equation}\label{eq:robschematic}
R_{tot} = \frac{l+d_1 R_1 + d_2 R_2 + \cdots}{l+d_1 + d_2 + \cdots}.
\end{equation}
Examining this expression noting that $R_i$ are all strictly less than one proves that networks maximizing $l$, will also maximize $R_{tot}$ (Supplementary Information).
This implies that the interaction graphs for systems that are the most robust will maximize the number of links between SCCs as well as the overall number of SCCs with respect to a particular system size. This analytical result predicts that any biological network whose associated dynamical system has the interaction graph \reffigscc $\,$ top will be more robust than those associated to any of the other interaction graphs in \reffigscc. Because this result is purely topological in nature, it does not depend at all upon any particular details such as the probability distribution from which the component interaction strengths are sampled or the size of the system. The result that dynamical robustness is correlated with system hierarchy therefore applies to an even broader class of dynamical systems than the particular random ensembles we have studied directly.

To test this prediction, we computed the probability distribution of stability and dynamical robustness relative to biological network architecture for ensembles of systems having two or three interacting components (see \ref{tab:structstabmat} and \ref{tab:structstabmat3}). For all of these, we found that robustness is correlated with connectivity, but that the most robust systems have intermediate connectivity for a given network size (\reffigrobustconnect). Accounting for the number of cycles in a network architecture reveals a strong correlation between robustness and connectivity that was hidden when networks with any number of cycles were considered together (\reffigconnectcycle3D3x3). While the most hierarchical network architecture will always lack cycles altogether, cycle number alone is clearly insufficient to account for robustness as the members of each class span nearly the entire range of possible robustness values. Consistent with our analysis of the symmetries of robustness, we found that the most hierarchical network architecture is the most robust (\reffigrobusthierarchy). Moreover, if we consider hierarchy partitioned by connectivity, we find that there is a monotonic increase in robustness following any line of increasing hierarchy in \reffigconnectdist3D3x3.

% \textcolor{red}{We precisely compute the probability of stability and the robustness as a distribution over system connectivity for all systems containing two or three interacting components. We find that stability to structural perturbations of this class of dynamical systems is correlated with connectivity, number of cycles, and the number of links between strongly connected components (SCCs) of the underlying interdependency graph. The latter correlation derives from the fact that the permutation of SCCs is a symmetry of dynamical robustness.}

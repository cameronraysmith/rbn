%!TEX root = ../paper.tex
% The analysis of dynamical systems that attempts to model chemical reaction, gene-regulatory, population, and ecosystem networks all rely on models having interacting components. When the details of these interactions are unknown for systems of interest, one effective approach is to study the dynamical properties of an ensemble of models determined by constraints that can be considered to apply to all such systems. Here we analyze the stability and robustness of a large class of dynamical systems. In particular, we precisely determine the probability distribution of robustness over system connectivity, which has significant implications from the study of metabolic to gene-regulatory to ecosystem network dynamics, for systems with two and three interacting components. We show that robustness is classified by the number of links between strongly connected components of the graph representing the underlying system connectivity leading to the conclusion that the most robust systems are also the most hierarchical. We also demonstrate that permutation of strongly connected components is a fundamental symmetry of dynamical robustness. This results in the classification of the dynamical robustness of biological networks based upon a purely topological property.

\begin{abstract}
The relationship between network topology and system dynamics has significant implications for unifying our understanding of the interplay among metabolic, gene-regulatory, and ecosystem network architecures. Here we analyze the stability and dynamical robustness of a large class of dynamics on such networks, the latter considered to occur over evolutionary timescales. We determine the probability distribution of dynamical robustness as a function of network topology and show that dynamical robustness is classified by the number of links between modules of the network. We also demonstrate that permutation of these modules is a fundamental symmetry of dynamical robustness. Analysis of these findings leads to the conclusion that the most dynamically robust systems correspond to the most hierarchical connectivity of these modules. This relationship provides a means by which evolutionary selection for a purely dynamical phenomenon may shape network architectures across scales of the biological hierarchy.
\end{abstract}

The analysis of dynamical systems that attempts to model chemical reaction, gene-regulatory, population, and ecosystem networks all rely on models having many parameters and thus many degrees of freedom. When the details of a system are unknown, one effective approach is to study the dynamical properties of a collection of models determined by constraints applying to all such systems. Here we analyze the stability of a large class of dynamical systems to perturbations in the underlying structure of the system: a property referred to as \emph{structural stability}. In particular, we precisely determine the probability distribution over system connectivity, a parameter which has significant implications from the study of gene-regulatory networks to ecosystem dynamics, of structural stability for systems with two and three interacting components. We show for these cases that structural stability has a non-monotonic relationship with system connectivity. We also demonstrate that networks with a hierarchical structure are more likely to be stable to perturbations than those with a more entangled heterarchical structure. These results support future work attempting to characterize the scaling relationship between structural stability and system connectivity. The latter investigation is necessary to evaluate several conjectured but ultimately untested hypotheses about biological networks.

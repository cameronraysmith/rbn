%!TEX root = ../paper.tex

In the general case of a dynamical system with $n$ components, where the components may be concentrations of chemical species, genes, or biological species, we have an $n$-dimensional vector of state variables or observables $(x_1(t), \ldots x_n(t)) = \vec{x}(t)
$
whose components are solutions to the arbitrary first order system
\begin{equation}\label{eq:eom}
\frac{dx_i(t)}{dt} = F_i(\vec{x}(t), \vec{p}), \; (i=1,\ldots,n)
\end{equation}
where $F_i$ represent, potentially nonlinear, functions of the given vector of state variables and $\vec{p}$ is the vector of parameters of the $F_i$. These parameters typically represent reaction rates in chemical and gene-regulatory networks, interaction strengths in ecological networks, or cooperativity exponents. For example, in the Lotka-Volterra model in \ref{fig:biomodelexamples}, $\vec{x} = (n_1, \ldots, n_N)$, $\vec{p}=(r_1,\ldots,r_N,b_{11},\ldots,b_{NN})$, and $F_i = r_i n_i + \sum_{j=1}^{N} b_{ij} n_i n_j$.

Fixed points are the simplest class of solutions to the dynamical system characterizing its long-term behavior. If $\vec x$ is a fixed point (i.e. $F_i(\vec{x})=0$ for all $i$), we
may proceed to ask whether it is dynamically stable.
Intuitively, dynamic stability means that, if one chooses the initial
conditions sufficiently close to the fixed point, the solution will
stay nearby.  Physically, this is important because,
if a fixed point ${\vec x}^0$ is unstable, we have zero probability of
observing the solution ${\vec x}(t) = {\vec x}^0$ in the absence of coupling to another system. The Lotka-Volterra model has two fixed points: the trivial one of all zero species $n_i=0$ and the other given implicitly by $r_i + \sum_{j=1}^{N} b_{ij} n_j = 0$.

To determine stability, we linearize the equations of motion (\refsupp{}) about the
fixed point $\vec{x}^0$:
\begin{equation}\label{eq:lineardynsys}
\frac{d\vec{y}(t)}{dt} = A \vec{y}(t),
\end{equation}
where $\vec{y} = \vec{x} - \vec{x}^0$ and the $n \times n$ Jacobian matrix $A$ has components
$$
a_{ij} = \left. \frac{\partial F_i}{\partial x_j} \right|_{\vec{x} = \vec{x}^0}.
$$
The system defined by $F_i$ and $\vec{x}^0$ is dynamically stable if the eigenvalues of $A$ all have real parts less than zero and $A$ is then referred to as a stable matrix (\refsupp{}). This criterion can be checked equivalently in terms of conditions on the coefficients of the characteristic polynomials $\chi(A)$ associated to the systems described by matrices $A$ \cite{Gantmacher1959}. As an example of a Jacobian matrix, the Lotka-Volterra model has
 \begin{equation}\label{eq:lotkavolterrajacobian}
   a_{ij} = \left\{
     \begin{array}{lr}
       r_i + b_{ii} n_i + \sum_{k=1}^{N} b_{ik} n_{k}, & i \neq j\\
       b_{ii} n_i & i=j
     \end{array}.
   \right.
\end{equation}

% \subsection{Overview}
The dynamical model given in terms of a system of differential equations for any network can be represented in terms of a global interaction graph (\ref{fig:biomodelexamples} top row).
For a general system the directed graph $G_X$ that describes the manner in which each of the variables depends upon one another is given by the adjacency matrix $\adj(G_X)$ where $\adj(G_X)_{ij}$ is $1$ for $F_i \hbox{ depends on } x_j$ and $0$ for $F_i \hbox{ does not depend on } x_j$. These two conditions on global system interactions are expressed respectively in terms of elements of the Jacobian matrix as $\frac{\partial F_i}{\partial x_j}(\vec{x}) \neq 0$ for all $\vec{x}$ and $\frac{\partial F_i}{\partial x_j}(\vec{x}) = 0$ for all $\vec{x}$.
%  \begin{displaymath}
%    \adj(G_X)_{ij} = \left\{
%      \begin{array}{ll}
%        1, & F_i \hbox{ depends on } x_j\\
%        0, & F_i \hbox{ does not depend on } x_j
%      \end{array}.
%    \right.
% \end{displaymath}
For large systems, since any given component is only likely to interact with a relatively small proportion of the other components, these matrices may be sparse.
We can also associate a local interaction graph $G_A$ given by an adjacency matrix $\adj(G_A)$ to each dynamical system having Jacobian matrix $A$ at some fixed point $\vec{x}^0$ where
 \begin{equation}\label{eq:lininterdepadj}
   \adj(G_A)_{ij} = \left\{
     \begin{array}{lr}
       1, & a_{ij} \neq 0\\
       0, & a_{ij} = 0
     \end{array}.
   \right.
\end{equation}
In general, the graph $G_A$ is a subgraph of $G_X$, however, $G_A$ is almost always equivalent to $G_X$ (\refsupp{}). We define the connectivity to be equal to the number of edges in $G_A$, which is equivalent to summing up the number of non-zero entries of $\adj(G_A)$.
These interaction graphs can be viewed as deriving from the combination of system components that accept a given pattern of inputs and produce a given pattern of outputs (\reffigexamplesystemmodules).

The network architecture represented in terms of the adjacency matrix, \ref{eq:lininterdepadj}, can be abstracted into modules by mapping the interaction graph to the network of strongly connected components (SCCs, see \refsupp{} and \reffigscc). This map from the interaction graph of a network to its SCCs, referred to as $\hier$ in \ref{fig:modsccsym}, has a collection of intrinsic symmetries shown in \reffighiertransformations{}. These three symmetries, \reffigscc{}, represent transformations that can be performed on the interaction graph that do not change the network of SCCs to which it is associated. Symmetries with respect to some property of the system are characterized by the ability to interchange these modules or their connectivity without changing that property. Two of these three intrinsic symmetries are also symmetries with respect to dynamical robustness. \ref{fig:robustnesssymmetries} shows an example of these latter symmetries applied to a specific interaction graph.

In the process of biological evolution, the parameter values $\vec{p}$ or the form of the functions $F_i$ corresponding to all different types of networks considered in \ref{fig:biomodelexamples} are subject to, potentially drastic, modifications due, for example, to mutations or environmental fluctuations. These changes may alter the capacity of a given system to achieve a stable steady state. In order to evaluate this we consider a stochastic process whereby for a given dynamical system associated to a stable matrix $A$, we consider a perturbation of the system leading to $A'$ where $\adj(G_A) = \adj(G_{A'})$ \ref{fig:robustnessconcept}. This corresponds to a dynamical system whose connectivity is preserved, but where the underlying model is defined in terms of a different vector of parameters ${\vec{p}}\,'$ (\refsupp{}). We then ask, for each distinct network topology on $n$ components, what is the probability, given $A$ is a stable matrix, that $A'$ is also a stable matrix. This probability is what we refer to as dynamical robustness.

We have derived an analytical expression for dynamical robustness, $R_{tot}$, of a network in terms of its interaction graph as a weighted average of the robustness of the SCCs, $R_i$, the corresponding number of links within each SCC, $d_i$, and the number of links between the SCCs, $l$. The entries of the Jacobian are sampled independently from a generic probability distribution. Examples of vector fields that correspond to this sampling of Jacobian matrices are shown in \ref{fig:jacobianvectorfields}. The expression in \ref{eq:sccrobustness} holds for the case of simultaneously perturbing any number of elements of the Jacobian, but can be schematized for the case of perturbing a single element as \ref{eq:robschematic} (see \reffigscc$\,$ for examples demonstrating this expression)
\begin{equation}\label{eq:robschematic}
R_{tot} = \frac{l+d_1 R_1 + d_2 R_2 + \cdots}{l+d_1 + d_2 + \cdots}.
\end{equation}
For instance, if our graph is the one in \reffigscc{} (middle panels), then we have two connected components, one with two nodes, and one with one node.  From \ref{tab:structstabmat}, we know that the graph with two nodes has probability $0.25$ of being stable and robustness $0.62$.  The graph with one node corresponds to a $1 \times 1$ matrix, so we have probability $0.5$ of stability and robustness $0.5$.  Thus, the probability of our 3-node graph being stable is $0.5 \times 0.25 = 0.125$ and its robustness is computed from \ref{eq:robschematic} in \reffigscc{}, which agrees with the value computed in \ref{tab:structstabmat} up to sampling error.

Examining this expression noting that $R_i$ are all strictly less than one proves that networks maximizing $l$, will also maximize $R_{tot}$.  Given two connected components $C_i$ and $C_j$ with $v_i$ and $v_j$ nodes respectively, we have a maximum of $v_i v_j$ links going from $C_i$ to $C_j$.  Hence, $l \le \sum_{(ij) \in \hier(G)} v_i v_j$.  Since every acyclic digraph can be embedded into a totally ordered
set, we may assume without loss of generality that our components have
been ordered in a way such that, if $(i,j) \in \hier(G)$, then $i <
j$.  Hence, $l \le l_{max}$ where
$$l_{max} = \sum_{i=1}^{n-1}\sum_{j=i+1}^{n}v_i
v_j~=~\frac{1}{2} \left( \sum_{i=1}^{n} v_i \right)^2-\frac{1}{2} \sum_{i=1}^{n}
v_i^2,$$
Hence, we conclude that $R_{tot} \le R_{max}$ where $R_{max}$ is equivalent to \ref{eq:robschematic} with $l_{max}$ substituted for $l$ and the upper bound is attained (\refsupp{}).

This argument also works when we resample more than one entry,
although the notation becomes more complicated.  Suppose that we
resample $m$ nodes.  Define
\begin{widetext}
\begin{equation*}
M = \left\{(m_0, m_1, \ldots, m_n) \,\bigg|\,
m = \sum_{i=0}^n m_i \quad\&\quad
m_0 \le \sum_{i=1}^{n-1} \sum_{j=i+1}^n \ell_{ij} \quad\&\quad
(\forall i \in \{1, \ldots, n\}) \; m_i < \connectivity_i \right\}.
\end{equation*}
\end{widetext}
Then, given $(m_0, m_1, \ldots, m_n) \in M$, there are ${m \choose
m_0, m_1, \ldots. m_n}$ ways of choosing $m_i$ links from $C_i$ and
$m_0$ links between strongly connected components.  Hence, our
weighted average becomes
\begin{widetext}
\begin{equation}
\langle R(\hbox{stab} (G), \mu') \rangle_m =
\frac{\sum\limits_{(m_0, m_1, \ldots m_n) \in M}
      {m \choose m_0, m_1, \ldots. m_n}
      \left(m_0 + \sum\limits_{i=1}^n m_i \langle R(\hbox{stab} (C_i), \mu') \rangle_{m_i} \right)}
     {\sum\limits_{(m_0, m_1, \ldots m_n) \in M}
      {m \choose m_0, m_1, \ldots. m_n} m}
\end{equation}
\end{widetext}

As before, since $\langle R(\hbox{stab} (C_i), \mu') \rangle_{m_i} \le 1$, we may
increase $\langle R(\hbox{stab} (G), \mu') \rangle_m$ by increasing the maximum
possible value of $m_0$ while keeping the strongly connected
components the same.  Again, if we fix $\hier(G)$, the maximum
possible value of $m_0$ is $\sum_{(i,j) \in \hier(G)} v_i v_j$
whereas, if we allow it to vary, the maximum is $\frac{1}{2} ((\sum_{i=1}^n
v_i)^2 - \sum_{i=1}^n v_i^2)$, which is attained when $\hier(G_{max}) =
G_{tot}$.  Hence, we conclude that $\langle R(\hbox{stab} (G), \mu') \rangle_m \le
\langle R(\hbox{stab} (G_{\mathrm{max}}), \mu') \rangle_m$.

This implies that the interaction graphs for systems that are the most robust will maximize the number of links between SCCs as well as the overall number of SCCs with respect to a particular system size. This analytical result predicts that any network whose associated dynamical system has the interaction graph equivalent to the total ordering like that of \reffigscc{} top will be more robust than those associated to any of the other interaction graphs in \reffigscc{}. Because this result is purely topological in nature, it does not depend at all upon any particular details such as the probability distribution from which the component interaction strengths are sampled or the size of the system. The result that dynamical robustness is correlated with network hierarchy therefore applies to an even broader class of dynamical systems than the particular random ensembles we have studied directly.

To test the prediction of this analytical result, we computed the probability distribution of stability and dynamical robustness relative to network architecture for ensembles of systems having two or three interacting components (see \ref{tab:structstabmat} and \ref{tab:structstabmat3}). For all of these, we found that robustness is correlated with connectivity, but that the most robust systems have intermediate connectivity for a given network size (\reffigrobustconnect). Accounting for the number of cycles in a network architecture reveals a strong correlation between robustness and connectivity that was hidden when networks with any number of cycles were considered together (\reffigconnectcycle3D3x3). While the most hierarchical network architecture will always lack cycles altogether, cycle number alone is clearly insufficient to account for robustness as the members of each class span nearly the entire range of possible robustness values. Consistent with our analysis of the symmetries of robustness, we found that the most hierarchical network architecture is the most robust (\reffigrobusthierarchy). Moreover, if we consider hierarchy partitioned by connectivity, we find that there is a monotonic increase in robustness following any line of increasing hierarchy in \reffigconnectdist3D3x3.

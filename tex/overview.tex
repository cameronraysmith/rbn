%!TEX root = ../paper.tex
\section{Dynamical Systems on Biological Networks}

To begin our analysis we will define dynamical systems on biological networks, review the manner in which their linear stability can be evaluated and describe how each one can be associated to a graph encoding interactions among system variables. In the general case of a dynamical system with $n$ variables, where the variables may be concentrations of chemical species, genes, or biological species, we have an $n$-dimensional vector of state variables or observables $(x_1(t), \ldots x_n(t)) = \vec{x}(t)
$
whose components are solutions to the arbitrary first order system
\begin{equation}\label{eq:eom}
\frac{dx_i(t)}{dt} = F_i(\vec{x}(t), \vec{p}), \; (i=1,\ldots,n)
\end{equation}
where $F=\{F_1,\ldots,F_i,\ldots,F_n\}$ represent, potentially nonlinear, functions of the given vector of state variables and $\vec{p} \in \mathbb{R}^{n'}$ is the vector of $n'$ parameters of the $F_i$. These parameters typically represent reaction rates or interaction strengths in chemical, gene-regulatory and ecological networks. For example, in the Lotka-Volterra model in \ref{fig:biomodelexamples}, $\vec{x} = (x_1, \ldots, x_n)$, $\vec{p}=(r_1,\ldots,r_n,b_{11},\ldots,b_{nn})$, and $F_i = r_i x_i + \sum_{j=1}^{n} b_{ij} x_i x_j$. The set of all dynamical systems, $\mathcal{D}'$, for a given number of state variables $n$ and a given number of parameters $n'$, is then
\begin{equation}\label{eq:setdynsys}
\mathcal{D}' = \{ (F(-,\vec{p}),\vec{p}) | F \colon \mathbb{R}^n \times \mathbb{R}^{n'} \rightarrow \mathbb{R}^n, \vec{p} \in \mathbb{R}^s \}.
\end{equation}
The space of all possible $F$ may be too large to precisely define probabilities over, in which case it is possible to consider various subspaces $\mathcal{D} \subset \mathcal{D}'$. In some cases these may even be given by parametrized models with a single fixed form of $F$ and then varying $\vec{p}$ alone would be sufficient to vary within that class of dynamical systems.

Fixed points are the simplest class of solutions to the dynamical system characterizing its long-term behavior. If $\vec x$ is a fixed point (i.e. $F_i(\vec{x})=0$ for all $i$), we
may proceed to ask whether it is dynamically stable.
Intuitively, dynamic stability means that, if one chooses the initial
conditions sufficiently close to the fixed point, the solution will
stay nearby.  Physically, this is important because,
if a fixed point ${\vec x}^0$ is unstable, we have zero probability of
observing the solution ${\vec x}(t) = {\vec x}^0$ in the absence of coupling to another system. The Lotka-Volterra model has two fixed points: the trivial one of all zero species $x_i=0$ and the other given implicitly by $r_i + \sum_{j=1}^{n} b_{ij} x_j = 0$. The set of all such fixed points, $\mathcal{F}$, for a collection of dynamical systems $\mathcal{D}$ is then
\begin{equation}\label{eq:setfixedpoints}
\mathcal{F} = \left\{ (d,\vec{x}^0) \; \left| \;
\begin{aligned}
d = (F(-,\vec{p}),\vec{p}) &\in \mathcal{D}, \\
\vec{x}^0 &\in \mathbb{R}^n,\\
F(\vec{x}^0,\vec{p}) &= 0
\end{aligned}
\right.
\right\}.
\end{equation}

% To determine stability, we linearize the equations of motion (\refsupp{}) about the
% fixed point $\vec{x}^0$:
% \begin{equation}\label{eq:lineardynsys}
% \frac{d\vec{y}(t)}{dt} = A \vec{y}(t),
% \end{equation}
% where $\vec{y} = \vec{x} - \vec{x}^0$ and the $n \times n$ Jacobian matrix $A$ has components
% $$
% a_{ij} = \left. \frac{\partial F_i}{\partial x_j} \right|_{\vec{x} = \vec{x}^0}.
% $$
\subsection{Stability analysis of biological networks}
To determine stability, we use the Taylor series expansion of the equations of motion, \ref{eq:eom}, about the fixed point $\vec{x}^0$ where $\vec{y} = \vec{x} - \vec{x}^0$ given by
\begin{equation}\label{eq:taylorseries}
\begin{aligned}
\frac{dx_i(t)}{dt} \approx & F_i(\vec{x}^0)
+ \sum_{j=1}^{N} \left. \frac{\partial F_i}{\partial x_j} \right|_{\vec{x} = \vec{x}^0} y_j\\
& + \frac{1}{2}\sum_{j,k=1}^{N} \left. \frac{\partial^2 F_i}{\partial x_j \partial x_k} \right|_{\vec{x} = \vec{x}^0} y_j y_k + \cdots
\end{aligned}.
\end{equation}
The zeroth order term vanishes since $F_i(\vec{x}^0)=0$ by definition and thus neglecting terms higher than first order from \ref{eq:taylorseries} results in
% \begin{equation*}
% \frac{d\vec{y}(t)}{dt} = A \vec{y}(t),
% \end{equation*}
\begin{equation}\label{eq:lineardynsys}
\frac{d\vec{y}(t)}{dt} = A \vec{y}(t),
\end{equation}
where the $n \times n$-matrix $A$ has components
$$
a_{ij} = \left. \frac{\partial F_i}{\partial x_j} \right|_{\vec{x} = \vec{x}^0}.
$$

The system defined by $F_i$, $\vec{p}$, and $\vec{x}^0$ is dynamically stable if the eigenvalues of $A$ all have real parts less than zero and $A$ is then referred to as a stable matrix. The spectral abscissa of the matrix $A$ is defined as
$$
\eta(A) = \max_i \{\Re(\lambda_i)\}
$$
where $\lambda_i$ are the eigenvalues of $A$. The system defined by $F_i$ and $\vec{x}^0$ is dynamically stable if the spectral abscissa of $A$ is less than zero, equivalently, $\eta(A) < 0$. This is because the general solution to \ref{eq:lineardynsys} is
$$
y_i(t) = \sum_j b_{ij} e^{\lambda_j t}, \; (i=1,\ldots,n)
$$
for some matrix $B=(b_{ij})$ and thus all $\vec{y} = \vec{x} - \vec{x}^0$ decay to zero when all $\lambda_i < 0$.

This criterion can be checked equivalently in terms of conditions on the coefficients of the characteristic polynomials $\chi(A)$ associated to the systems described by matrices $A$. In the $2$-dimensional case, $\chi(A) = \lambda^2 + Tr(A)\lambda+Det(A)$ has solutions $\lambda$ with negative real parts if $Tr(A)<0$ and $Det(A)>0$, which we make use of in examples. Generalized conditions for higher dimensions are available in \cite{Gantmacher1959}. As an example of a Jacobian matrix, the Lotka-Volterra model has
 \begin{equation}\label{eq:lotkavolterrajacobian}
   a_{ij} = \left\{
     \begin{array}{lr}
       r_i + b_{ii} x_i + \sum_{k=1}^{n} b_{ik} x_{k}, & i \neq j\\
       b_{ii} x_i & i=j
     \end{array}.
   \right.
\end{equation}

Evaluation of the stability criterion occurs on a space of two states inducing a mapping from matrices to binary values $\mathcal{S} \colon \mathbb{R}^{n \times n} \rightarrow \{ 1, 0 \}$ given by
 \begin{equation}\label{eq:stabeval}
   \mathcal{S}(A) = \left\{
     \begin{array}{lr}
       1, & \eta (A) < 0\\
       0, & \eta (A) \geq 0
     \end{array},
   \right.
\end{equation}
where $1$ stands for $S$ or stable and $0$ stands for $U$ or unstable. The stability criterion defines an equivalence relation on the set of all Jacobian matrices $A \in \mathbb{R}^{n \times n}$ deriving from fixed points on $n$ variables that simply splits the set into two classes $S=\{ A \, | \, A \hbox{ is stable}  \}$ and $U~=~\{ A \, | \, A \hbox{ is unstable} \}$.

\subsection{Equivalence classes of systems associated to Jacobian Matrices}
We would ultimately like to determine the probability of a stable system remaining stable in the context of evolution modeled as a stochastic process operating on a large set of dynamical systems. Any given Jacobian matrix, which is what characterizes the stability of a system near a given fixed point, does not refer to a single dynamical system, but rather a set of fundamentally distinct dynamical systems from which that Jacobian matrix could have derived. More precisely, matrices of numbers corresponding to some Jacobian matrix evaluated at some fixed point define an equivalence relation, $\sim$, on the set of fixed points, $\mathcal{F}$, given by $(F,\vec{p},\vec{x}^0) \sim (F',\vec{p}\,',\vec{x}^{0'}) $ if and only if
\begin{equation}\label{eq:jaceqrel}
\left. \frac{\partial F(\vec{x},\vec{p})}{\partial \vec{x}} \right|_{\vec{x} = \vec{x}^0} =
\left. \frac{\partial F'(\vec{x},\vec{p}\,')}{\partial \vec{x}} \right|_{\vec{x} = \vec{x}^{0'}}.
\end{equation}
This relation then partitions the set of all fixed points into equivalence classes $\mathcal{F}/{\sim}$, where the class $[A]$ associated to Jacobian matrix $A \in \mathbb{R}^{n \times n}$ is
\begin{equation}\label{eq:jaceqs}
[A] = \left\{ (F,\vec{p},\vec{x}^0) \; \Bigg| \; \left. \frac{\partial F(\vec{x},\vec{p})}{\partial \vec{x}} \right|_{\vec{x} = \vec{x}^0} = A \right\}.
\end{equation}
An example with two members of $\mathcal{F}$ from each of four different equivalence classes $[A_1]$, $[A_2]$, $[A_3]$, and $[A_4]$ of $\mathcal{F}/{\sim}$ is shown in \ref{fig:robustnessconcept}.

% \subsection{Overview}
\subsection{Interaction graphs encoding network architecture}
We are interested in the manner in which system variables interact with one another, and, ultimately, whether dynamical robustness depends upon the topology of these interactions. Interactions among system variables can be encoded in a directed graph associated to the system. Strictly speaking, the interactions among variables in a dynamical model for any network can be represented in terms of a global interaction graph (\ref{fig:biomodelexamples} top row).
For a general system $X \in \mathcal{D}$ the directed graph $G_X$ that describes the manner in which each of the variables depends upon one another can be represented by the adjacency matrix $\adj(G_X)$ where $\adj(G_X)_{ij}$ is $1$ if $F_i \hbox{ depends on } x_j$ and $0$ if $F_i \hbox{ does not depend on } x_j$. These two conditions on global system interactions are expressed locally for all $\vec{x}$ in terms of elements of the Jacobian matrix as $\frac{\partial F_i}{\partial x_j}(\vec{x}) \neq 0$ and $\frac{\partial F_i}{\partial x_j}(\vec{x}) = 0$.
%  \begin{displaymath}
%    \adj(G_X)_{ij} = \left\{
%      \begin{array}{ll}
%        1, & F_i \hbox{ depends on } x_j\\
%        0, & F_i \hbox{ does not depend on } x_j
%      \end{array}.
%    \right.
% \end{displaymath}
For large systems, since any given variable is only likely to interact with a relatively small proportion of the other variables, these matrices may be sparse.
We can also associate a local interaction graph $G_A$ given by an adjacency matrix $\adj(G_A)$ to each dynamical system having Jacobian matrix $A$ at some fixed point $\vec{x}^0$ where
 \begin{equation}\label{eq:lininterdepadj}
   \adj(G_A)_{ij} = \left\{
     \begin{array}{lr}
       1, & a_{ij} \neq 0\\
       0, & a_{ij} = 0
     \end{array}.
   \right.
\end{equation}
In general, the graph $G_A$ is a subgraph of $G_X$, however, for the classes of systems of interest here, $G_A$ is equal to $G_X$, and thus $\adj(G_A) = \adj(G_X)$ (see \refsupp{} \ref{sec:reactionnetjacobian}). We define the connectivity to be equal to the number of edges in $G_A$, which is equivalent to summing up the number of non-zero entries of $\adj(G_A)$.
These interaction graphs can be viewed as deriving from the combination of system components that accept a given pattern of inputs and produce a given pattern of outputs (\reffigexamplesystemmodules).

Each distinct labeled directed graph $G$ (of which there are $k=2^{n(n-1)}$) that could be associated to the interactions in a model defined on $n$ variables, selects a subset of $\mathcal{F} / {\sim}$ (see~\ref{fig:robustnessprocess}C)
\begin{equation}\label{eq:jacgrapheqs}
[G] = \left\{ [A] \; | \; G_A = G \right\}.
\end{equation}
The $G$-classes thereby partition the collection of fixed points, $\mathcal{F}$, over the space of dynamical systems, $\mathcal{D}$, according to the interactions among the variables of the dynamical system represented by the topology of $G$. This partition, $\mathcal{F} / G$, is a coarsening of $\mathcal{F} / {\sim}$. In what follows we consider the stability of classes of fixed points associated to each possible graph $G$. In most cases some of these are stable and some of them are unstable, which we quantify as the probability of stability. However, note that for some $G$, all fixed points respecting the topology of $G$ are unstable.

\section{Evolutionary processes sampling dynamical systems}
In the course of biological evolution, the parameter values $\vec{p}$, form of the functions $F$, and environmental conditions restricting access to the basins associated to different fixed points $\vec{x}^0$ corresponding to all different types of networks considered in \ref{fig:biomodelexamples} are subject to, potentially drastic, modifications due to environmental fluctuations. The stochastic process by which these modifications occur induces a process on the set of fixed points that results in the assignment of a probability $P(f^T)$ to each history of length $T$, $f^T = ( f_1,f_2,\ldots,f_T ) \in \mathcal{F}^T$ \cite{RobertM.Gray130}.
% $$
% P(f',t+1 | f,t) = \mathcal{T}_{\mathcal{F}}(f',f)
% $$
% for all $f,f' \in \mathcal{F}$.
These dynamics induce a stochastic process on the equivalence classes of fixed points $\mathcal{F}/{\sim}$ indexed by Jacobian matrices given by
$$
P([A]^T) = \sum_{ \{ f^T \mid f_i^T \in [A]_i^T \} } P(f^T)
$$
for each history of length $T$, $[A]^T = ( [A_1], [A_2], \ldots, [A_T] ) \in (\mathcal{F}/{\sim})^T$
% $$
% P([A'],t+1 | [A],t) = \mathcal{T}_{\mathcal{F}/{\sim}}([A'],[A])
% $$
% for all $A,A' \in \mathcal{F}/{\sim}$
as visualized in \ref{fig:robustnessprocess}B. These changes can alter the stability of a given system, thereby inducing an even more coarse-grained stochastic process on the stable regions of the space of fixed points given by
$$
P(s^T) = \sum_{ \{ A^T \mid \mathcal{S}(A_i^T) = s_i^T \} } P(A^T)
$$
for each history of length $T$, $s^T = ( s_1, s_2, \ldots, s_T ) \in \{0,1\}^T$ where $s=1 \equiv S,\, s=0 \equiv U$ as visualized in \ref{fig:robustnessprocess}A and B. In order to model this, we consider a process whereby perturbations applied to a given dynamical system and fixed point associated to a stable Jacobian matrix $A$ lead to another Jacobian matrix $A'$.  This corresponds to an ensemble of fixed points of dynamical systems where each model in the ensemble may otherwise be defined in terms of a different collection of rate functions $F'$, vector of parameters ${\vec{p}}\,'$, or environmental conditions restricting access to $\vec{x}^{0'}$ (\refsupp{} \ref{sec:reactionnetjacobian}). We then ask what is the probability, given $A$ is a stable matrix, that $A'$ is also a stable matrix. This quantifies the intuitive statement that, over evolutionary timescales, it is not enough for a system to be stable, but rather that it have a reasonably high probability of remaining stable over contiguous timeframes. In terms of a history $s^T$ over the states specifying the stability property, $\mathcal{S}$, this is given by
$$
r = \frac{\sum_{t=0}^{T-1} s_{t+1}^T s_{t}^T}{\sum_{t=0}^{T-1} s_{t}^T}.
$$
If the process $P(s^T)$ has limits $\mu(s) = \lim_{t,T \rightarrow \infty} P(s_t^T=s)$ and $\tau(s' \mid s) = \lim_{t,T \rightarrow \infty} P(s_{t+1}^T = s' \mid s_t^T = s)$ for $s,s' \in \{0,1\}$
%stationary probability $P^{\infty}(s)$ for $s \in \mathcal{S}$
, the expectation of $r$ as $T \rightarrow \infty$ is approximated by
\begin{equation}\label{eq:robustnessfromstoch}
R(S,\mu,\tau) = E[r] \approx \frac{\sum_{s,s'} \mu(s) \tau(s' \mid s)}{\sum_{s=1} \mu(s)}.
\end{equation}
This conditional probability corresponds to the parameter $R$ of the two-state process depicted in \ref{fig:robustnessprocess}A, and it is what we refer to as dynamical robustness.

Conceptually, one might consider the quantity defined by \ref{eq:robustnessfromstoch} to be related to structural stability \cite{Smale1967}, but its definition differs in several important ways (see \refsupp{} \ref{sec:dynrobustandstructstab}). Briefly, structural stability is a qualitative, global property that distinguishes dynamical systems whose associated vector field topologies are invariant to \emph{any} sufficiently small perturbation from those whose topologies change for \emph{some} perturbation of any size. Dynamical robustness, on the other hand, is a quantitative, local property that specifies the probability of the stability of a fixed point undergoing a transformation from stable to unstable when the system is subjected to, potentially large, perturbations in one or a few interactions among system variables.

In terms of Jacobian matrices $A$ and $A'$, if the underlying process $P([A]^T)$ has analogous limits
$$\mu(A) = \lim_{t,T \rightarrow \infty} P(A_t^T=A)$$
and
$$\tau(A' \mid A) = \lim_{t,T \rightarrow \infty} P(A_{t+1}^T = A' \mid A_t^T = A)$$
for $A,A' \in \mathbb{R}^{n \times n}$, \ref{eq:robustnessfromstoch} becomes
\begin{equation}\label{eq:robustnessonjacobians}
R(S,\mu,\tau) = \frac{\int_{A,A' \in \mathbb{R}^{n \times n}} d\mu(A) d\tau(A' \mid A) \mathcal{S}(A) \mathcal{S}(A')}{\int_{A \in \mathbb{R}^{n \times n}} d\mu(A) \mathcal{S}(A)}.
\end{equation}

We would like to determine whether selection for dynamical robustness would result in a preference for the maintenance of one kind of network architecture relative to another. In order to then classify the properties of this process according to network architecture, we consider cases where the condition $\adj(G_A) = \adj(G_{A'})$ holds, which results in analogous processes defined on each $G$-class $[G] \in \mathcal{F}/G$ given by
$$
P([A]^T \mid G_A {=} G) = \sum_{ \{ f^T \mid f_i^T \in [A]_i^T \} } P(f^T)
$$
for each history of length $T$, $[A]^T = ( [A_1], [A_2], \ldots, [A_T] ) \in [G]^T$
as visualized in \ref{fig:robustnessprocess}C. This now corresponds to a collection of ensembles of dynamical systems each with equivalent connectivities corresponding to a graph $G$. For each graph, there is a potentially different value of robustness given by
\begin{equation}\label{eq:robustnessbygraph}
\begin{aligned}
R(G,S,\mu,\tau)  =
 \frac{\int\limits_{A,A' \in \mathbb{R}^{n \times n} \atop G_A = G} d\mu(A) d\tau(A' \mid A) \mathcal{S}(A) \mathcal{S}(A')}{\int\limits_{A \in \mathbb{R}^{n \times n} \atop G_A = G} d\mu(A) \mathcal{S}(A)}.
\end{aligned}
\end{equation}
Comparing the values of $R$ for each $G$-class in $\mathcal{F}/G$ places a partial ordering on network architectures, $G$, which allows for the determination of which network architectures would be expected to be enriched relative to others in an evolutionary process where selection is imposed in a manner that results in a bias toward higher $R$ values.

\section{Properties of network architecture}

\subsection{Abstracting network architectures to their strongly connected components}
We will ultimately find that it is possible to decompose the computation of dynamical robustness from \ref{eq:robustnessfromstoch} over a certain kind of modular structure intrinsic to any network architecture. Here we introduce the manner in which such a decomposition can be consistently performed for any architecture. The network architecture represented in terms of the adjacency matrix, \ref{eq:lininterdepadj}, can be abstracted into modules by decomposing the interaction graph into its associated network of strongly connected components (SCCs). A SCC of a graph is a maximal subset of nodes where each node within the subset can be reached from any other in some number of (not necessarily single) steps \cite{Cormen2009}. The SCCs of some examples of three variable systems are outlined in \reffigscc{} along with their adjacency matrices.

The map from the interaction graph of a network to its SCCs, referred to as $\hier$ in \ref{fig:modsccsym}, results in a decomposition of $G$ into its SCCs. Each node of $\hier (G)$ corresponds to a SCC of $G$. There is an edge from the node corresponding to component $C$ to the node corresponding to component $C'$ if and only if there exists a link from some node in $C$ to some node in $C'$ in $G$. Note that $\hier (G)$ is acyclic for any $G$ because of the maximality property of SCCs.

One can also perform this construction in the opposite direction to produce one of the graphs consistent with a given specification of SCCs and their connectivity.  Start with a directed acyclic graph $H$.  To each node $n$ of $H$ associate
a strongly connected graph $C_n$.  To each link $(i,j)$ of $H$ associate a non-empty subset of $\Node(C_i) \times \Node(C_j)$.  The result will be a graph $G$ such that $\hier(G) = H$ and furthermore, every graph $G$ such that $\hier(G) = H$ can be obtained in this manner.

\subsection{Quantification of network hierarchy}

The relationship between $G$ and $\hier(G)$ for all $G$ with a given number of nodes suggests a heuristic method of quantifying the degree of hierarchy of a given graph and thus of the system structure it represents. The most hierarchical system is considered to be the graph corresponding to the total ordering (\refsupp{} and \cite{Cormen2009}), which for three nodes is given in \reffigscc{} (top panel). This graph possesses the maximum number of SCCs for any graph with a given number of nodes, $n$, as well as the maximum number, $\binom{n}{2}$, of links between SCCs. The graph edit distance (ED) on a fixed number of nodes from one graph to another is defined as the minimum number of modifications of the first graph in order to transform it into the second \cite{Axenovich2011}. This distance between any given graph and the total ordering thus quantitatively represents how far a graph is from being maximally hierarchical. In this work we take $max(ED) - ED$ to be the definition of hierarchy, where $max(ED)$ is the maximum edit distance for all graphs with a given number of nodes.

\subsection{Symmetries of SCCs and their relationship to robustness}
This map $\hier$ is many-to-one and so there is a large class of
operations which leaves $\hier(G)$ invariant for a given graph $G$, \reffighiertransformations{}. These three symmetries, \reffigscc{}, represent transformations that can be performed on the interaction graph that do not change the network of SCCs to which it is associated.
For instance, we may interchange the positions of the SCCs relative to each other \reffighiertransformations$a$.  Leaving the components fixed, we may move links between nodes in a component \reffighiertransformations$b$ or between components, or even add or delete links \reffighiertransformations$c$.

Symmetries with respect to some property of the system are characterized by the ability to interchange these modules or their connectivity without changing that property. Two of these three intrinsic symmetries of $\hier$ are also symmetries with respect to dynamical robustness. \ref{fig:robustnesssymmetries} shows an example of these latter symmetries applied to a specific interaction graph. The collections of networks that have equivalent dynamical robustness characterized by its symmetries allow for a classification of neutral networks with respect to any evolutionary process selecting for dynamical robustness.

\section{Derivation of the relationship between network architecture and robustness}
Here we show that dynamical robustness is maximized by the most hierarchical of network architectures. In doing so, we restrict to a set of networks that has the same SCCs but may have different number and connectivity of links between the components, consider a stochastic process as a simple model for evolution that independently resamples individual elements of Jacobian matrices, and maximize robustness in that context.

We demonstrate that the Jacobian matrix associated to a system at a fixed point is stable in the sense of \ref{eq:stabeval} if and only if each of the blocks corresponding to its SCCs are all independently stable. We derive an analytical expression for dynamical robustness, $R$, of a network in terms of its interaction graph, $G$, as a weighted average of the dynamical robustness values, $R_{\alpha}$, of the SCCs, $C_{\alpha}$, the corresponding number of links within each SCC, $d_{\alpha}$, and the number of links between the SCCs, $l$ to give a total connectivity $d = l + \sum_{\alpha} d_{\alpha}$. For a given collection of SCCs, we show that the network architectures maximizing dynamical robustness must maximize the number of links between those SCCs. The latter corresponds to the most hierarchical network architecture that can be constructed from that set of SCCs.
%The network architecture having $l=l_{max}$ is the most hierarchical one. Comparing the robustness of a network architecture having $l = l_{max}$ to any other network architecture having any other value of $l$, we show that the network architecture having $l = l_{max}$ maximizes dynamical robustness.
These results ultimately hold for those processes involving simultaneous perturbation of any number of elements of the Jacobian. To anchor the intuition before stating the more general result, we derive the expression for the case of perturbing a single element at each timestep of the stochastic process being used as a simplistic model of evolution.

\subsection{Decomposition of stability over SCCs}
We begin by expressing the stability of a network in terms of the stabilities of its SCCs. Let the index $i$ range over the non-zero entries $a_i$ of $A$. The entries of the Jacobian are sampled independently from a generic probability distribution $\rho_i$ for each entry. Under this assumption then
\begin{equation}\label{eq:singleresamplemutau}
\begin{aligned}
\mu(A) &= \prod_i \rho_i(a_i),\\
\tau^{(1)}(A' \mid A) &= \frac{1}{d} \sum_i \tau^{(1)}_i (A' \mid A),
\end{aligned}
\end{equation}
where
$$
\tau^{(1)}_i(A' \mid A) = \rho_i(a'_i) \prod_{j \neq i} \delta(a'_j - a_j).
$$

The decomposition of a digraph into SCCs corresponds to a block triangular decomposition of its adjacency matrix.  Say that the graph $G$ has SCCs $C_1, C_2, \ldots C_n$, where we now use $n$ with no index to specify the number of SCCs, which have been labelled in such a way that there are no links from nodes in component $C_i$ to component $C_j$ when $i < j$.  Label the nodes in such a way that $V_1, \ldots, V_{n_1}$ belong to $C_1$, $V_{n_1 + 1}, \ldots, V_{n_2}$ belong to $C_2$, etc.  Then, if we choose basis vectors corresponding to this labelling of the nodes, we will have $a_{ij} = 0$ whenever $i$ and $j$ correspond to different components and $i > j$.  This condition is equivalent to stating that the matrix is block triangular with blocks of size $n_1, n_2, \ldots$.

Since the determinant of a triangular matrix equals the product of the determinants of its diagonal blocks, it follows that the characteristic polynomial factors as the product of the charactericstic polynomials of its diagonal blocks.  Hence, a block triangular matrix is stable if and only if its diagonal blocks are stable.  Note that this condition does not depend upon the entries off the diagonal (which correspond to links between SCCs) and does not depend upon what order the components appear. This fact implies that the terms evaluating stability such as $\mathcal{S}(A)$ from \ref{eq:robustnessbygraph} decompose into products over the SCCs
\begin{equation}\label{eq:stabfactor}
\mathcal{S}(A) = \prod_{C_\alpha \in \hier(G)} \mathcal{S}(\pi_{C_{\alpha}}(A))
\end{equation}
where $\pi_{C_{\alpha}}(A)$ denotes the projection of the matrix $A$ onto the SCC $C_{\alpha}$.

\subsection{General computation of dynamical robustness}

To relate the robustness of a graph to the robustness of its SCCs, we substitute \ref{eq:singleresamplemutau} and \ref{eq:stabfactor} into \ref{eq:robustnessbygraph}, collect factors corresponding to components, decompose integrals into their respective products, collapse integrals over delta distributions, and cancel common factors between numerator and denominator.  Let $L$ denote the set of edges of $G$ that connect distinct SCCs.  Then, if $i \in C_\alpha$ for some SCC $C_\alpha$, we have
\begin{widetext}
\begin{equation}\label{eq:robustncessforsccs}
\begin{aligned}
\frac{\int d\mu(A) d\tau^{(1)}_i (A' \mid A) \mathcal{S}(A) \mathcal{S}(A')}
       {\int d\mu(A) \mathcal{S}(A)}
&= \frac{\begin{matrix}
  \int \prod\limits_{k \in C_\alpha} da_k\,d{a'}_k\, \rho(a_k)
    \rho_i({a'}_i) \prod\limits_{j \in C_\alpha \setminus i} \delta ({a'}_j - a_j)
    \mathcal{S}(\pi_{C_{\alpha}}(A)) \mathcal{S}(\pi_{C_{\alpha}}(A')) \times \\
 \prod\limits_{C_\beta \in \hier(G) \setminus C_\alpha} \int
   \prod\limits_{j \in C_\beta} da_k\,d{a'}_k\, \rho({a'}_j) \delta ({a'}_j - a_j)
      \mathcal{S}(\pi_{C_{\beta}}(A)) \mathcal{S}(\pi_{C_{\beta}}(A')) \times \\
 \prod\limits_{l \in L} \int da_l\,d{a'}_l\,\rho({a'}_l) \delta({a'}_l - a_l) \end{matrix}}
{\begin{matrix}\int \prod\limits_{k \in C_\alpha} da_k\,\rho(a_k) \mathcal{S}(\pi_{C_{\alpha}}(A)) \times \\
 \prod_{C_\beta \in \hier(G) \setminus C_\alpha}
   \int \prod\limits_{h \in C_\alpha} da_h,\rho(a_h) \mathcal{S}(\pi_{C_{\beta}}(A)) \times \\
 \prod\limits_{l \in L} \int da_l\,\rho(a_l) \end{matrix}} \\[1em]
&= \frac{\int \prod\limits_{k \in C_\alpha} da_k\,d{a'}_k\,\rho(a_k)
    \rho_i({a'}_i) \prod\limits_{j \in C_\alpha \setminus i} \delta ({a'}_j - a_j)
    \mathcal{S}(\pi_{C_{\alpha}}(A)) \mathcal{S}(\pi_{C_{\alpha}}(A'))}
{\int \prod\limits_{k \in C_\alpha} dk\, \rho(a_k) \mathcal{S}(\pi_{C_{\alpha}}(A))} \\
&= R(C_{\alpha}, S, \mu_\alpha, \tau^{(1)}_\alpha)
\end{aligned}
\end{equation}
\end{widetext}
where $\mu_\alpha$ and $\tau^{(1)}_\alpha$ refer to the analogues of \ref{eq:singleresamplemutau} for the subgraph $C_\alpha$ of $G$. \ref{eq:robustncessforsccs} shows that the terms in the sum over the elements of $A$, or equivalently the edges of $G$, that comprise a given SCC of $G$ reduce to the robustness of that SCC alone.  Likewise, when $i \in L$, we have
\begin{widetext}
\begin{equation}\label{eq:robustncessforsccconnect}
\begin{aligned}
\frac{\int d\mu(A) d\tau^{(1)}_i (A' \mid A) \mathcal{S}(A) \mathcal{S}(A')}
       {\int d\mu(A) \mathcal{S}(A)}
&= \frac{\begin{matrix}
  \prod\limits_{C_\alpha \in \hier(G)} \int
   \prod\limits_{j \in C_\beta} da_k\,d{a'}_k\, \rho({a'}_j) \delta ({a'}_j - a_j)
      \mathcal{S}(\pi_{C_\alpha}(A)) \mathcal{S}(\pi_{C_\alpha}(A')) \times \\
 \int da_i d{a'}_i \rho_i(a) \rho_a({a'}_i) \times
 \prod\limits_{l \in L\setminus i} \int da_l\,d{a'}_l\,\rho({a'}_l) \delta({a'}_l - a_l) \end{matrix}}
{\prod_{C_\alpha \in \hier(G)}
   \int \prod\limits_{h \in C_\alpha} da_h,\rho(a_h) \mathcal{S}(\pi_{C_\alpha}(A)) \times
 \int da_i \rho(a_i) \times
 \prod\limits_{l \in L \setminus i} \int da_l\,\rho(a_l)} \\
&= 1
\end{aligned}
\end{equation}
\end{widetext}
For each $C_{\alpha} \in \hier(G)$, there will be $d_\alpha$ values of $i$ such that $i \in C_{\alpha}$ requiring instances of \ref{eq:robustncessforsccs}; likewise, there will be $l$ values of $i$ such that $i \in L$ requiring instances of \ref{eq:robustncessforsccconnect}.  Hence, when we perform the summation over $i$ to compute the robustness of $\mathcal{F}/G$ for each $G$, we will obtain a weighted average:
\begin{equation}\label{eq:robschematic}
R = \frac{l+d_1 R_1 + d_2 R_2 + \cdots}{l+d_1 + d_2 + \cdots}.
\end{equation}
Here $R_\alpha$ is shorthand for $R(C_\alpha,S,\mu_\alpha,\tau^{(1)}_\alpha)$ and $R$ is shorthand for $R(G,S,\mu,\tau^{(1)})$.

\subsection{Example computation of robustness}
Examples of vector fields that correspond to the sampling of Jacobian matrices used in the computation of robustness for particular examples are shown in \ref{fig:robustnessconcept} and \ref{fig:jacobianvectorfields}. \ref{eq:robschematic} shows a schematized version of \ref{eq:robustnessbygraph} for the case of perturbing a single element at a time  (see \reffigscc$\,$ for examples demonstrating this expression). For instance, if our graph is the one in \reffigscc{} (middle panels), then we have two connected components, one with two nodes, and one with one node.  From \ref{tab:structstabmat}, we know that the graph with two nodes has probability $0.25$ of being stable and robustness $0.62$.  The graph with one node corresponds to a $1 \times 1$ matrix, so we have probability $0.5$ of stability and robustness $0.5$.  Thus, the probability of our 3-node graph being stable is $0.5 \times 0.25 = 0.125$ and its robustness is computed from \ref{eq:robschematic} in \reffigscc{}, which agrees with the value computed in \ref{tab:structstabmat} up to sampling error.

\subsection{Most hierarchical network architectures maximize dynamical robustness}
Here we show that dynamical robustness is maximized for a given collection of SCCs by maximizing the number of links between them. This is accomplished by demonstrating that any graph $G$ with fewer than the maximum, $l_{\mathrm{max}}$, of links between the given collection of SCCs must have a lower dynamical robustness than the graph, $G_{\mathrm{max}}$, with $l=l_{\mathrm{max}}$ and that this $G_{\mathrm{max}}$ is indeed the graph $G_{\mathrm{tot}}$ which gives the total ordering on the SCCs. Examining the expression in \ref{eq:robschematic} and noting that $R_{\alpha}$ are all strictly less than one proves that networks maximizing $l$, will also maximize $R$.  Given two connected components $C_{\alpha}$ and $C_{\beta}$ with $v_{\alpha}$ and $v_{\beta}$ nodes respectively, we have a maximum of $v_{\alpha} v_{\beta}$ links going from $C_{\alpha}$ to $C_{\beta}$.  Hence, $l \le \sum_{(\alpha,\beta) \in \hier(G)} v_{\alpha} v_{\beta}$.  Since every acyclic digraph can be embedded into a totally ordered
set, we may assume without loss of generality that our components have
been ordered in a way such that, if $(\alpha,\beta) \in \hier(G)$, then $\alpha <
\beta$.  Hence, $l \le l_{\mathrm{max}}$ where
$$l_{\mathrm{max}} = \sum_{\alpha=1}^{n-1}\sum_{\beta=\alpha+1}^{n}v_{\alpha}
v_{\beta}~=~\frac{1}{2} \left( \sum_{\alpha=1}^{n} v_{\alpha} \right)^2-\frac{1}{2} \sum_{\alpha=1}^{n}
v_{\alpha}^2.$$
Suppose we have a graph $G$ with SCCs $C_1,\ldots,C_n$ and that $G_{\mathrm{tot}}$
is the graph on $n$ nodes with a link from node $i$ to node $j$ whenever $i < j$.  Then we have a graph $G_{\mathrm{max}}$ such that $\hier (G_{\mathrm{max}}) = G_{\mathrm{tot}}$, the components of $G_{\mathrm{max}}$ are also $C_1, \ldots, C_n$, but that includes all possible links between each pair of components. In this case $R(G)~\le~R(G_{\mathrm{max}})$ for all $G \ne G_{\mathrm{max}}$ where $R(G_{\mathrm{max}})$ is equivalent to \ref{eq:robschematic} with $l_{\mathrm{max}}$ substituted for $l$
\begin{widetext}
\begin{align} \label{eq:sccmaxrobustness}
R(G_{\mathrm{max}}) =
\frac{\frac{1}{2} ((\sum_{\alpha=1}^n v_{\alpha})^2 - \sum_{\alpha=1}^n v_{\alpha}^2) +
                    \sum_{\alpha=1}^n \connectivity_{\alpha}
                    % \langle R(\hbox{stab}(C_i),\mu') \rangle_1
                R(C_{\alpha},S,\mu_{\alpha},\tau_{\alpha})}
     {\frac{1}{2} ((\sum_{\alpha=1}^n v_{\alpha})^2 - \sum_{\alpha=1}^n v_{\alpha}^2) +
                    \sum_{\alpha=1}^n \connectivity_{\alpha}}.
\end{align}
\end{widetext}
% where $R(G_{\mathrm{max}},S,\mu,\tau) = R_{\mathrm{max}}(C_1, \ldots C_n)$.
%This bound is attained by $G_{\mathrm{max}}$.
% $\langle R(\hbox{stab} (G_{\mathrm{max}}), \mu') \rangle_1 = R_{\mathrm{max}}(C_1, \ldots C_n)$.
\subsection{Simultaneous perturbation of multiple interaction strengths}
This argument also works when relationships between multiple system variables are perturbed simultaneously,
although the notation becomes more complicated.  Suppose that we
resample $m$ interactions.  Then the analogue of \ref{eq:singleresamplemutau} is
\begin{equation}\label{eq:multipleresamplemutau}
\begin{aligned}
\tau^{(m)}(A' \mid A) &= \frac{1}{\binom{d}{m}} \sum_{i_1, \ldots, i_m}
  \tau^{(m)}_{i_1, \ldots, i_m} (A' \mid A), \\
\tau^{(m)}_{i_1, \ldots, i_m} (A' \mid A) &=
 \prod_{k=1}^m \rho_{i_k} (a'_{i_k})
 \prod_{j \notin \{i_1, \ldots i_m\}} \delta(a'_j - a_j).
\end{aligned}
\end{equation}
Now define
\begin{widetext}
\begin{equation*}
M = \left\{(m_0, m_1, \ldots, m_n) \,\bigg|\,
m = \sum_{i=0}^n m_i \quad\&\quad
m_0 \le \sum_{i=1}^{n-1} \sum_{j=i+1}^n \ell_{ij} \quad\&\quad
(\forall i \in \{1, \ldots, n\}) \; m_i < \connectivity_i \right\}.
\end{equation*}
\end{widetext}
Then, given $(m_0, m_1, \ldots, m_n) \in M$, there are ${m \choose
m_0, m_1, \ldots. m_n}$ ways of choosing $m_i$ links from $C_i$ and
$m_0$ links between SCCs.  Hence, our
weighted average becomes
\begin{widetext}
\begin{equation}\label{eq:robustnessmultiple}
R(G, S, \mu, \tau^{(m)}) =
\frac{\sum\limits_{(m_0, m_1, \ldots m_n) \in M}
      {m \choose m_0, m_1, \ldots. m_n}
      \left(m_0 + \sum\limits_{\alpha=1}^n m_\alpha R(C_\alpha, S, \mu, \tau^{(m_\alpha)}) \right)}
     {\sum\limits_{(m_0, m_1, \ldots m_n) \in M}
      {m \choose m_0, m_1, \ldots. m_n} m}
\end{equation}
\end{widetext}

As before, since $R(C_\alpha, S, \mu, \tau^{(m_i)}) \le 1$, we may
increase $R(G, S, \mu, \tau^{(m)})$ by increasing the maximum
possible value of $m_0$ while keeping the SCCs the same.  Again, if we fix $\hier(G)$, the maximum
possible value of $m_0$ is $\sum_{(\alpha,\beta) \in \hier(G)} v_\alpha v_\beta$
whereas, if we allow it to vary, the maximum is $\frac{1}{2} ((\sum_{\alpha=1}^n
v_\alpha)^2 - \sum_{\alpha=1}^n v_\alpha^2)$, which is attained when $\hier(G_{max}) =
G_{tot}$.  Hence, we conclude that $R(G, S, \mu, \tau^{(m)}) \le
R(G_{\mathrm{max}}, S, \mu, \tau^{(m)})$.

This analytical result implies that the interaction graphs for systems that are the most robust will maximize the number of links between a given set of SCCs. Because this result is purely topological in nature, it does not depend at all upon any particular details such as the probability distribution from which the variable interaction strengths are sampled or the size of the system. Any network whose associated dynamical system has the interaction graph among its SCCs equivalent to the total ordering is more robust than those associated to any of the other interaction graphs in \reffigscc{}. The graph associated to the total ordering is the most hierarchical network architecture for any given number of SCCs like that of \reffigscc{} top for three variable systems where the \emph{highest} variable in the hierarchy has directed links to all other variables in the network, the second highest variable has directed links to all other variables in the network except the highest one, \emph{et cetera} (\refsupp{}). The result that dynamical robustness is correlated with network hierarchy therefore applies to an even broader class of dynamical systems than the particular random ensembles we have studied directly.

\subsection{Examples of the analytical relationship between hierarchy and robustness}

To test the prediction of the analytical results in \ref{eq:robschematic} and \ref{eq:robustnessmultiple}, we computed approximations to the probability distribution of stability and dynamical robustness relative to network architecture for ensembles of systems having two or three interacting variables (see \refsupp{} \ref{tab:structstabmat} and \ref{tab:structstabmat3}). For all of these, we found that robustness is correlated with connectivity, but that the most robust systems have intermediate connectivity for a given network size (\reffigrobustconnect). Accounting for the number of cycles in a network architecture reveals a strong correlation between robustness and connectivity that was hidden when networks with any number of cycles were considered together (\reffigconnectcycle3D3x3). While the most hierarchical network architecture will always lack cycles altogether, cycle number alone is clearly insufficient to account for robustness as the members of each class span nearly the entire range of possible robustness values. Consistent with our analysis of the symmetries of robustness, we found that the most hierarchical network architecture is the most robust (\reffigrobusthierarchy). Moreover, if we consider hierarchy partitioned by connectivity, we find that there is a monotonic increase in robustness following any line of increasing hierarchy in \reffigconnectdist3D3x3.

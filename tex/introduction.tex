%!TEX root = ../paper.tex
The traditional approach taken in the study of chemical reaction, gene-regulatory, population, and ecosystem networks is to derive a system of differential equations to model a particular biological network, attempt to fit that model to data and adjust the modeling assumptions along with parameter values until a good fit is achieved \cite{Meyer2014}. All of these models utilize essentially equivalent mathematical structures \ref{fig:biomodelexamples} \cite{RossCr2003,Alon2006,Palsson2006,HamidBolouri2008,Palsson2011a,Voit2012,Sauro2012}. Developing unified mathematical descriptions of all of these is one of the paramount goals of systems biology.

Recent work has demonstrated that as a result of inhomogeneity in the geometry of the sensitivity of the dynamics over the parameter space for such models, this approach allows for a large variety of models to fit the data \cite{Brown2003,Gutenkunst2007,Daniels2008a,Machta2013,Hines2014,Prabakaran2014,Tonsing2014}. In addition, there is often uncertainty about the very structure of such biological networks. In this context, it is crucial to gain insight into what dynamical phenomena are possible to observe within a given class of dynamical systems, which is necessary to understand in order to determine whether or not a particular dynamical phenomenon should be regarded as unique or generic in the development and investigation of models applied to particular systems \cite{Gunawardena2013,Gunawardena2014}. This can be achieved using a method common in statistical physics involving the consideration of an ensemble of systems that, in comparison to one another, appear to have components that are randomly interlinked.

Indeed, investigating generic properties of a large class of dynamical systems was the approach taken by May in models of ecosystem dynamics \cite{Gardner1970,May1972}. The class of dynamical systems studied by May is, however, not restricted to ecosystem dynamics and encompasses, among others, the dynamics of all of the biological networks represented in \ref{fig:biomodelexamples}. May conjectured on the basis of results from random matrix theory what eventually came to be referred to as the May-Wigner stability theorem \cite{Cohen1984,May1972a,Radius2014}, which implies a relationship between a topological property, system connectivity, and a dynamical property, stability.


% The generic applicability of gaining a better understanding of the class of models investigated by May demonstrates the unequivocal value of deeper investigation.  However, work attempting to continue the development of the so-called May-Wigner stability theorem revealed that May's conjectured stability criterion was not as easy to demonstrate as was initially believed.
